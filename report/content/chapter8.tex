%----------------------------------------------------------------------------
\chapter{Összefoglalás}
%----------------------------------------------------------------------------

Diplomamunkámban két rögzített kamera videofolyamai alapján előállítottam a megfigyelt térrészben mozgó objektumok rekonstrukcióját egy, a két kamera között választott nézőpontból. Áttekintettem a gépi látás szakirodalmában publikált és a feladatomhoz kapcsolódó legfontosabb eredményeket, különös tekintettel azok elméleti hátterére és az alkalmazható algoritmusokra. Megterveztem és elkészítettem egy szimulációs alkalmazást, aminek segítségével az eljárást három jeleneten teszteltem. A szoftver segítségével a használt kamerákat bekalibrálhatjuk, és azok videofolyamai alapján folyamatos rekonstrukciót végezhetünk, melyhez tartozó nézőpontot egy csúszka segítségével változtathatjuk a két kamera között. A valós idejű feldolgozás korlátait megvizsgáltam, az eredmények alapján a megoldást a gyakorlatban is alkalmazhatónak találtam.

Az alkalmazott kamerák számának további növelésével a nézőpont helyzete, ahonnan megfelelő rekonstrukcióra számíthatunk, nagyobb szabadságfokkal választható. Például egy kellő számú kamerával felszerelt teremben mozgó személyek helyzetei a valós idejű videofolyamok alapján nagyon jó közelítéssel meghatározhatóak, amely egy biztonsági megoldás alapjául szolgálhat.

A továbbfejlesztés egy másik lehetősége a feldolgozási teljesítmény növelése, mely hatékonyabb algoritmusokkal és azok célhardveren történő futtatásával érhető el.
