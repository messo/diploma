{\color{blue}

%----------------------------------------------------------------------------
\chapter{Összefoglalás}
%----------------------------------------------------------------------------

A feladatkiírás pontjainak eleget tettem. Áttekintettem a szükséges irodalmat, elméleti hátteret, valamint megterveztem a megoldáshoz szükséges alkalmazást. Leírtam az implementálás részleteit, a hozott tervezői döntéseket, valamint kitértem a választott algoritmusok indoklására. Az elkészült alkalmazást két jeleneten teszteltem, a valós idejű feldolgozás korlátait megvizsgáltam, valamint részletesen dokumentáltam az eredményeket.

Az 1. fejezetben a feladat értelmezését követően a szükséges lineáris algebrai kitekintő után az alapvető háromdimenziós transzformációkat tárgyaltam. Végül a szakirodalomban és a dolgozat során is használt lyukkamera-modellel zártam, mely a kép- és videofeldolgozási algoritmusok alapjául szolgál.

A 2. fejezetben a több kamerából álló rendszereknél felmerülő, a feladatom során is megoldandó problémákat jártam körbe. Kitértem a kamerák szinkronizációjára, valamint a kamerák által látott képekből a háromdimenziós világ lehetséges rekonstrukciójára. Ehhez két lényegében eltérő megoldást vázoltam és mutattam be; a sztereó-kalibrációt valamint az optikai folyamokat. A fejezetet az objektum-detektálással zártam, mely felhasználásával a probléma megoldásának keresési tere csökkenthető. Az első két fejezettel a feladatkiírásom első pontjának eleget tettem.

A 3. fejezetben a célkitűzésem megfogalmazása után az előző fejezetekben megismert megközelítésekre és algoritmusokra támaszkodva megterveztem az elkészítendő rendszert.

A 4. fejezetben a megvalósítás lépéseit írtam le, az alkalmazás struktúráját osztály- és szekvencia diagramokkal szemléltettem, valamint az egyes fázisok eredményeit ábrákkal demonstráltam. Kitértem a megoldáshoz használt OpenCV-s függvényekre, valamint ezek paraméterezéseire.

Az 5. fejezetben az elkészült alkalmazást két jelenet segítségével teszteltem, melyek során két eltérő kamera telepítést alkalmaztam. Az eredményeket ábrákon keresztül bemutattam, valamint véleményt alkottam ennek minőségéről. Ezzel és az előző két fejezettel a feladatkiírásom második, valamint részben az utolsó pontjának tettem eleget.

A 6. fejezetben megvizsgáltam, majd alkalmaztam néhány lehetőséget, mellyel a teljesítmény javítható, ezeket mérésekkel is alátámasztottam. Végül megvizsgáltam, hogy milyen korlátok mellett, milyen sebesség érhető el. Az eredmények alapján azt találtam, hogy választható olyan kicsi, de még hasznos információval rendelkező képrészlet, mely esetén a közel valós idejű feldolgozás megvalósítható. Ezzel a feladatkiírásom utolsó két pontjának feleltem meg.

A 7. fejezetben tömören összesítem és értékelem az elért eredményeket, valamint kitérek a továbbfejlesztési lehetőségekre.

}