%----------------------------------------------------------------------------
\appendix
%----------------------------------------------------------------------------
\chapter*{\fuggelek}\addcontentsline{toc}{chapter}{\fuggelek}
\setcounter{chapter}{6}  % a fofejezet-szamlalo az angol ABC 6. betuje (F) lesz
\setcounter{equation}{0} % a fofejezet-szamlalo az angol ABC 6. betuje (F) lesz
\numberwithin{equation}{section}
\numberwithin{figure}{section}
\numberwithin{lstlisting}{section}
%\numberwithin{tabular}{section}

%----------------------------------------------------------------------------
\section{Az alkalmazás elérhetősége}
%----------------------------------------------------------------------------

Az elkészült alkalmazás elérhető online egy Git tárolóban az alábbi címen: 

\url{https://github.com/messo/diploma}

Az alkalmazás C++ nyelven íródott, a platform független támogatást és a függőségek kezelését a CMake teszi lehetővé. Linuxon az alkalmazás a következő parancsok kiadásával fordítható:

\begin{lstlisting}[language=bash,basicstyle=\ttfamily\small]
   $ mkdir build && cd build
   $ cmake ..
   $ make -j4
\end{lstlisting}

A folyamat végén 3 bináris készül: \texttt{Calibration}, \texttt{PoseCalculation} és \texttt{Diploma}. Az elsővel generálhatóak a kamerák belső paramétereit tartalmazó fájlok, másodikkal a fix telepítésű kamerák elhelyezéseit számolhatjuk ki, majd a harmadik az előzőek által generált fájlok alapján a folyamatos helyreállítást végzi.
