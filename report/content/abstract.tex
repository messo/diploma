%\pagenumbering{roman}
%\setcounter{page}{1}

%----------------------------------------------------------------------------
% Abstract in Hungarian
%----------------------------------------------------------------------------
\selectlanguage{magyar}
\hungarianParagraph

{\color{blue}
\chapter*{Kivonat}\addcontentsline{toc}{chapter}{Kivonat}
}

A szakirodalomban jól ismert probléma egy jelenet háromdimenziós helyreállítása különböző pozíciókba helyezett kamerák által készített képek alapján, melyet utána szabadon bejárhatunk, több nézőpontból megfigyelhetünk.

%A diplomamunkámban megvizsgálom ennek a problémának a térrészben mozgó objektumokra korlátozott valós idejű megoldhatóságát. Két rögzített kamera képeiből és azok pozíciójából a mozgó objektumokat három dimenzióban a két kamera között választott nézőpontból rekonstruálom, melyhez egy szimulációs szoftvert készítettem. Bemutatom az alkalmazás tervezési és implementációs lépéseit, valamint konkrét példákon keresztül demonstrálom a helyreállítást két különböző elrendezésben telepített kamerákkal.

A diplomamunka e problémakört járja körbe, megvizsgálja adott térrészben mozgó objektum(ok)ra vonatkozóan a valós idejű megoldhatóság feltételeit és korlátait. Leírja két rögzített kamera videojelfolyamából és a kamerák pozíciójából kiindulva a mozgó objektum(ok) három dimenziós rekonstrukciójának lépéseit egy, a két kamera között választott nézőpontból. A feladat során erre a célra készített szimulációs szoftver segítségével konkrét példákon keresztül demonstrálja a helyreállítást két, különböző elrendezésben telepített kamera párral.

%Végül mérésekkel támasztom alá a rendelkezésre álló eszközökön elért teljesítményt, mely alapján a megoldás kisebb felbontású képeken közel-valós idejűnek tekinthető.

Az elkészült implementációval elért teljesítményt a rendelkezésre álló eszközökön mérésekkel támasztja alá, melyek alapján a megvalósított megoldás adott felbontás mellett közel valós idejűnek tekinthető.

\vfill


%----------------------------------------------------------------------------
% Abstract in English
%----------------------------------------------------------------------------
\selectlanguage{english}
\englishParagraph

{\color{red}
\chapter*{Abstract}\addcontentsline{toc}{chapter}{Abstract}
}

\vfill


%-----------
% RESET
% ----------

\dolgozatnyelve
\defaultParagraph

\newcounter{romanPage}
\setcounter{romanPage}{\value{page}}
\stepcounter{romanPage}