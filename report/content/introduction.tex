%----------------------------------------------------------------------------
\chapter{Bevezető}
%----------------------------------------------------------------------------

Az informatika, ezen belül pedig a gépi látással foglalkozó terület, valamint az ehhez szükséges számítási kapacitás és célhardverek rohamos fejlődésével mind újabb, hatékonyabb valamint pontosabb megoldások születtek és születnek a felmerült problémákra.

Míg egy-két évtizeddel ezelőtt a gépi látáshoz kapcsolódó kutatások jelentős részét főként a robotika, a katonság (pl. drónok), valamint az űrkutatás adta, manapság már a mindennapi élet gyökeres részévé vált. Vegyük például a közlekedést: a közép-felső kategóriás autóknál már szériatartozéknak tekinthető az elülső és hátsó tolató radar. Ugyanígy a kereskedelmi forgalomban kapható robot-porszívók is rendelkeznek beépített kamera/radar-rendszerrel, amely a beltéri navigációt segíti. A napjainkban kapható játékkonzolokhoz is vásárolható kiegészítő kamera rendszer, mely a játékos mozgását és pozícióját figyeli, lényegében a játékos a saját testét használja vezérlőként. Szórakozást tekintve a manapság egyre nagyobb teret kapó quadcoptereket \cite{quadropter} is említeni kell, ezek is rendelkeznek kamerával (vagy rájuk szerelhető), és már folynak kutatások, amelyek ezek akár autonóm \cite{quad-autonomous}, akár tömeges \cite{quad-swarm} -- rajban történő -- repülését vizsgálja.

A dolgozat motivációját a következő feltevésekhez hasonló problémák adták:
\begin{itemize}
\item Egy izgalmas gólhelyzet során mit láthatott a kapus a kapuban állva?
\item Egy bankrablási szituációban mit láthatott az elkövető, és mit a biztonsági őr?
\item Egy vizsga során, a gyanús egyetemista láthatta-e az előtte ülő dolgozatát?
\item Egy közlekedési balesetben mit láthatott a biciklis, és mit a buszsofőr?
\end{itemize}

Ezekre és ehhez hasonló kérdésekre részben választ nyújthat a kitűzött feladat megoldása, miszerint több kamerával megfigyelt térrészt egy választott nézpontból rekonstruálunk, mivel a fenti esetekben nem oldható meg, hogy a választott személy nézőpontjába valódi kamerákat állítsunk.

A diplomamunka részletesen leírja azt az eljárást, amely két rögzített kamera videofolyamai alapján előállítja a mozgó objektumok rekonstrukcióját egy választott nézőpontból. Az egyszeri kalibrációt követően a videófolyamok képkockáit több lépésben dolgozza fel. Először a mozgó objektumokat azonosítja két fázisban. Elsőként kijelöli azon képrészleteket, melyek az előtérhez tartoznak, majd utána ezeket párosítja a kamerák képein jellegzetes pontok segítségével. Az így adódó képrészlet párosítások adják az ugyanazon valódi objektumhoz tartozó képrészleteket a bementi képkockákon. Ezt követően az optikai folyamok segítségével az objektumok két képkockához tartozó pontjai között sűrű pont-pont megfeleltetést ad. A kamerák helyzeteit felhasználva a pontpárokból háromszögeléssel meghatározza a pontok háromdimenziós koordinátáit, melyek alapján végül rekonstruálja a választott nézőpontból látható képet.

A 2. fejezetben a feladat értelmezését követően a megoldásához szükséges elméleti háttérről lesz szó, mely tartalmazza a háromdimenziós tér alapvető transzformációit, valamint a dolgozat során használt kameramodellt. A 3. fejezet a több kamerából álló kamerarendszereknél felmerülő problémákat tárgyalja, illetve, hogy ezekre milyen, a napjainkban is használt megoldások léteznek, kitérve a feladat során szükséges részproblémák megoldására is. A 4. fejezet bemutatja az előző fejezetek által leírt információk és algoritmusok alapján egy, a kitűzött feladatra megoldást adó rendszer vázát, megvalósítási tervét. A 5. fejezetben a rendszer megvalósítáról, és az aközben hozott tervezői döntésekről lesz szó. Végigvezet az egyes fázisoknál elkészült implementációkon, és végül bemutatja a választott eljárásokat. Az 6. fejezet az elkészült alkalmazás segítségével két különböző helyzetben telepített kamerákkal felvett jelenetet rekonstruál. A 7. fejezet a gyorsítási lehetőségeket vizsgálja meg, kitér a párhuzamosítás lehetőségeire, valamint a GPU-n történő futtatásra, és az így elérhető sebességnövekedésre. Végül megvizsgálja a valós idejű helyreállíthatóság korlátait, melyeket konkrét számokkal támaszt alá. Végezetül a 8. fejezet a gyűjtött eredményeket foglalja össze, és értékeli az elkészült alkalmazást.
