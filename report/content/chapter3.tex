%----------------------------------------------------------------------------
\chapter{Tervezés \label{chapter3}}
%----------------------------------------------------------------------------

Az elméleti megfontolások, alapelvek bemutatását követően ebben a fejezetben rátérek a feladat megoldását adó rendszer struktúrájának, főbb komponenseinek tárgyalására.

\section{Specifikáció kidolgozása}

A diplomamunka során elkészített rendszernek képesnek kellett lennie egy zárt térrész tetszőlegesen választott pontjában és irányában látható, a mozgó objektumokat tartalmazó kép helyreállítására fix telepítésű kamerák valós idejű videofolyamai alapján. A kamerák fix helyzete egyszerűsíti a problémát, mert ezzel a kamera koordinátarendszerét kalibráció segítségével előre össze lehetett hangolni a világ-koordinátarendszerrel. A helyreállítást akkor tekintettem sikeresnek, ha a mozgó objektumokat detektáltam és választott nézőpontból azok megközelítő kontúrjait sikeresen meghatároztam. A tényleges objektum struktúrájának illetve textúrájának meghatározása nem esett a dolgozat hatáskörébe. A választott nézőpontra is voltak korlátaink, érdemi rekonstrukciót csak a kamerákat összekötő képzeletbeli szakaszon és annak környezetében várhatunk, így a tesztelést is így végeztem el.

A rendszer tervezése során figyelembe vettem, hogy lényegében tetszőleges számú kamerát is felhasználhattam a probléma megoldásához, de az elérhető eszközök korlátozott száma miatt csak két kamerát felhasználva készült el a konkrét implementáció.

Egyre több kamera bevonásával a rekonstrukció minősége, valamint a helyesen rekonstruálható nézőpontok száma növekszik, de ezzel együtt a rendszer teljesítménye a megnövekedő számítási szükséglet miatt csökken. Fontos megemlíteni, hogy a feladat és a megoldás jellegéből adódóan a független kamerákból meghatározott rekonstrukciók párhuzamosíthatóak, így ezek megfelelő feldolgozó egységet feltételezve egy időben elvégezhetőek. Ez azt jelenti, hogy a kamerák számának növelése valóban megoldás lehet a robusztusabb eredmény eléréséhez.

\section{Keretrendszer}

A diplomamunka készítése során az OpenCV \cite{opencv} keretrendszert használtam, melynek célja, hogy a fejlesztőknek egy szabadon és ingyenes elérhető alkalmazás-könyvtárat biztosítson a gépi látás területén elterjedt és gyakran használt algoritmusokhoz. Több nyelvhez is biztosít API-t, én ezek közül a C++-os interfészét alkalmaztam, a lehető legnagyobb teljesítmény elérése érdekében. Az OpenCV-t már az előző félévekben megismertem, így a diplomamunka során már eredményesen tudtam építkezni az általa nyújtott funkciókra.

\section{A tervezett rendszer működése}

A feladat megoldásához \aref{methods:optic}. részben leírt optikai folyam alapú eljárást használtam fel. A rendszer tervét, átfogó képét a következőkben mutatom be.

A kamerákat két lépésben kalibráltam; először, hogy a torzításukat minimalizáljam, majd azért, hogy meghatározhassam az elhelyezkedésüket egy rögzített koordinátarendszerben, melyet egy sakktábla segítségével definiáltam. Ezt \aref{fig:plan-calibration}. ábra szemlélteti.

\begin{figure}[tbh]
\centering

\begin{tikzpicture}[->,>=stealth',shorten >=1pt,auto]
\tikzset{
box/.style={draw, rectangle, text width=10em, minimum height=3em, text centered, fill=white},
plain/.style={text width=10em, text centered},
line/.style = {-,shorten >=0pt},
graybox/.style = {box, gray}
}

\node[box,copy shadow,text width=13.5em] (PicsForDistortion) {Képek gyűjtése több különböző pozícóban egy sakktábláról};

\node[box,text width=10em] (Distortion) [right of=PicsForDistortion,xshift=12em] {Kamera torzításának meghatározása};

\node[box,text width=8em] (Chessboard) [below of=Distortion,yshift=-3em] {Kép egy rögzített sakktábláról};

\node[box,text width=8em] (Pose) [right of=Chessboard,xshift=8em] {Kamera helyzetének meghatározása};

\draw (PicsForDistortion) -- (Distortion);

\draw (Distortion) -- (Chessboard);

%\draw[->,color=black,rounded corners=2mm] (Distortion.east) -| ++(1.5em,-3em) -|
%                 ([xshift=-1.5em]Chessboard.west) -- (Chessboard.west);

\draw (Chessboard) -- (Pose);

\end{tikzpicture}

\caption{Kamerák kalibrációja \label{fig:plan-calibration}}
\end{figure}

Ezt követően a kamerák képeiből meghatároztam az előtér maszkot. Erre kétféle megközelítést is implementáltam; az egyik az optikai folyamokat használja a mozgás érzékeléséhez, ami kijelöli az előteret, a másik pedig egy előtér-háttér szegmentációs eljárás. Ezeket \aref{fig:plan-mask}. ábra mutatja be, közülük összehasonlítás után választottam, lásd a következő fejezet.

\begin{figure}[tbh]
\centering

\begin{tikzpicture}[->,>=stealth',shorten >=1pt,auto]
\tikzset{
box/.style={draw, rectangle, text width=10em, minimum height=3em, text centered, fill=white},
plain/.style={text width=10em, text centered},
line/.style = {-,shorten >=0pt},
graybox/.style = {box, gray}
}


\node[plain] (input1) {képkocka};

\node[plain] (input2) [right of=input1,xshift=14em] {képkocka};


\node[box] (Model) [below of=input1,yshift=-3em] {Előtér-háttér modell építése};

\node [box] (OptFlow) [below of=input2,yshift=-1em] {Optikai folyam gyors számolása};

\node[plain,text width=4em] (prev) [right of=OptFlow,xshift=7em] {előző\\ képkocka};


\draw (input1) -- (Model);

\draw (input2) -- (OptFlow);
\draw (prev.west) -- (OptFlow);

\node [box] (Moving) [below of=OptFlow,yshift=-2.5em] {Folyam alapján mozgó részek meghatározása};

\draw (OptFlow) -- (Moving);

\node [box] (PostProcess) [below of=Moving,yshift=-4em,xshift=-8.25em] {Maszk\\ utófeldolgozása};

\draw (Model) -- (PostProcess) node [midway, left] (mask1) {előtér maszk};
\draw (Moving) -- (PostProcess) node [midway, right] (mask2) {előtér maszk};

\node (l2) [above of=PostProcess] {};
\node (l1) [above of=PostProcess,yshift=14em] {};

\draw[line,dashed,gray] (l1) -- (l2);

\end{tikzpicture}

\caption{Előtér maszk meghatározásának két módja \label{fig:plan-mask}}
\end{figure}


Miután a vélt előtér objektumok maszkjait meghatároztam, ezek által kijelölt képrészleteket (\textit{blob}ok) a kamerák képein fel kellett ismernem és egymással párosítanom. Az első egy egyszerű ,,legnagyobb-területű'' kiválasztás, amely mindkét képen a legnagyobb területű \textit{blob}ot keresi, és ezeket párosítja egymással. Természetesen ezzel csak egy objektumot rekonstruálhatunk, de a további lépéseket jól lehet rajta tesztelni. A másik eljárás pedig, hogy a képrészleteken jellegzetes pontokat kerestem, ezeket egymással párosítottam, majd többségi döntést alkalmazva meghatároztam, hogy az egyik kép blobja melyik másik blobnak felel meg. Így megkaptam az ugyanazon objektumokhoz tartozó képrészleteket a kamerák képein. Ezt a két folyamatot \aref{fig:plan-objects}. ábra mutatja be.


Minden objektumhoz külön-külön optikai folyamokkal sűrű pont-pont megfeleltetést számoltam a két képen. Felhasználva, hogy a kamerák helyzetét ismertem, háromszögeléssel megkaptam az adott objektum mindkét képen látható pontjainak világbeli koordinátáit.


\begin{figure}[b]
\centering

\begin{tikzpicture}[->,>=stealth',shorten >=1pt,auto]
\tikzset{
box/.style={draw, rectangle, text width=10em, minimum height=3em, text centered, fill=white},
plain/.style={text width=10em, text centered},
line/.style = {-,shorten >=0pt},
graybox/.style = {box, gray}
}


\node[box,text width=7em] (Largest) {Legnagyobb\\ blobok\\ kijelölése\\ és párosítása};

\node[plain,text width=6.5em] (input1_1) [left of=Largest,xshift=-8em,yshift=2em] {bal kép\\ előtér maszkja};

\node[plain,text width=6.5em] (input1_2) [left of=Largest,xshift=-8em,yshift=-2em] {jobb kép\\ előtér maszkja};

\node[plain,text width=4em] (output1) [right of=Largest,xshift=6em] {egyetlen\\ objektum};

\draw (input1_1) -- (Largest);
\draw (input1_2) -- (Largest);
\draw (Largest) -- (output1);

% ---------------------

\node[plain,text width=4em] (input2_1) [below of=input1_1,yshift=-8em] {bal kép előtér\\ maszkja};

\node[box,text width=6em] (Feature1) [right of=input2_1,xshift=5em] {Jellegzetes\\ pontok\\ detektálása\\ és leírása};

\node[plain,text width=4em] (input2_2) [below of=input2_1,yshift=-4em] {jobb kép előtér\\ maszkja};

\node[box,text width=6em] (Feature2) [right of=input2_2,xshift=5em] {Jellegzetes\\ pontok\\ detektálása\\ és leírása};

\draw (input2_1) -- (Feature1);
\draw (input2_2) -- (Feature2);

\node[box,text width=6em] (Pairing) [right of=Feature1,xshift=6em,yshift=-3.25em] {Leírók párosítása};

\draw (Feature1) -- (Pairing);
\draw (Feature2) -- (Pairing);

\node[box,text width=7em] (BlobMatching) [right of=Pairing,xshift=6em] {Leíró-párok alapján\\ blobok\\ párosítása};

\draw (Pairing) -- (BlobMatching);

\node[plain,text width=4em] (output2) [right of=BlobMatching,xshift=5em] {több\\ objektum};

\draw (BlobMatching) -- (output2);

\node (l1) [above of=input2_1,xshift=-4em,yshift=1.75em] {};
\node (l2) [right of=l1,xshift=37em] {};


\draw[line,dashed,gray] (l1) -- (l2);

\end{tikzpicture}

\caption{Objektumok detektálásának két módja \label{fig:plan-objects}}
\end{figure}



Végül a választott nézőpontból projekció segítségével meghatároztam az onnan látható becsült képet (felhasználva az eredeti képből kinyerhető színinformációkat), valamint az objektumokhoz tartozó kontúrokat.

A fenti eljárás tetszőleges számú kamera-párra (amelyek nem feltétlen diszjunktak) általánosítható, és az így kapott pontfelhők -- felhasználva, hogy az adatokat az előzetes kalibrációnak köszönhetően egy közös világ-koordinátarendszerben kapjuk meg -- könnyen egyesíthetőek. A diplomaterv keretében \aref{fig:of-method}. ábrán (\pageref{fig:of-method}. oldal) látható színezett hátterű lépések készültek el, a több kamerára vonatkozó együttes kezelés nem.

\begin{sidewaysfigure}
\centering

\pgfdeclarelayer{background}
\pgfdeclarelayer{foreground}
\pgfsetlayers{background,main,foreground}

\resizebox{\textwidth}{!}{%
\begin{tikzpicture}[->,>=stealth',shorten >=1pt,auto]
\tikzset{
box/.style={draw, rectangle, text width=12em, minimum height=2.5em, text centered},
plain/.style={text width=12em, text centered},
line/.style = {-,shorten >=0pt},
graybox/.style = {box, gray}
}

\node[plain] (Cam1) {1. kamera};

\node[graybox,fill=green!10] (Calib1) [below of=Cam1] {Kalibrálás};

\node[box,fill=green!10] (GetFrames1) [below of=Calib1,yshift=-1.5em] {Aktuális képkocka lekérése};

\node[box,fill=green!10] (FgMask1) [below of=GetFrames1,yshift=-1.5em] {Előtér maszkok meghatározása};

\draw (Calib1) -- (GetFrames1);
\draw (GetFrames1) -- (FgMask1);



\node[plain] (Cam2) [right of=Cam1,xshift=14em] {2. kamera};

\node[graybox,fill=green!10] (Calib2) [below of=Cam2] {Kalibrálás};

\node[box,fill=green!10] (GetFrames2) [below of=Calib2,yshift=-1.5em] {Aktuális képkocka lekérése};

\node[box,fill=green!10] (FgMask2) [below of=GetFrames2,yshift=-1.5em] {Előtér maszkok meghatározása};

\draw (Calib2) -- (GetFrames2);
\draw (GetFrames2) -- (FgMask2);



\node[plain] (Cam3) [right of=Cam2,xshift=14em] {3. kamera};

\node[graybox] (Calib3) [below of=Cam3] {Kalibrálás};

\node[box] (GetFrames3) [below of=Calib3,yshift=-1.5em] {Aktuális képkocka lekérése};

\node[box] (FgMask3) [below of=GetFrames3,yshift=-1.5em] {Előtér maszkok meghatározása};

\draw (Calib3) -- (GetFrames3);
\draw (GetFrames3) -- (FgMask3);



\node[plain] (CamN) [right of=Cam3,xshift=18em] {$n$. kamera};

\node[graybox] (CalibN) [below of=CamN] {Kalibrálás};

\node[box] (GetFramesN) [below of=CalibN,yshift=-1.5em] {Aktuális képkocka lekérése};

\node[box] (FgMaskN) [below of=GetFramesN,yshift=-1.5em] {Előtér maszkok meghatározása};

\draw (CalibN) -- (GetFramesN);
\draw (GetFramesN) -- (FgMaskN);


\node[plain] (LDots) [right of=GetFrames3,xshift=7.75em] {$\ldots$};


\node[box,fill=green!10] (Obj12) [below of=FgMask1,xshift=8.25em,yshift=-2.5em] {Objektumok detektálása a képpárokon};
 
\draw (FgMask1) |- (Obj12);
\draw (FgMask2) |- (Obj12);


\node[box] (Obj23) [below of=FgMask2,xshift=8.25em,yshift=-2.5em] {Objektumok detektálása a képpárokon};
 
\draw (FgMask2) |- (Obj23);
\draw (FgMask3) |- (Obj23);


\node[box,fill=green!10] (OF12) [below of=Obj12,yshift=-1.5em] {Optikai folyam meghatározása};
 
\draw (Obj12) -- (OF12);

\node[box] (OF23) [below of=Obj23,yshift=-1.5em] {Optikai folyam meghatározása};
 
\draw (Obj23) -- (OF23);



\node[plain] (OF34) [below of=FgMask3,xshift=6em,yshift=-2.5em,text width=2em] {$\ldots$};
\draw (FgMask3) |- (OF34);

\node[plain,below of=FgMaskN,xshift=-6em,yshift=-2.5em,text width=2em] (OFnn) {$\ldots$};
\draw (FgMaskN) |- (OFnn);



\node[box,fill=green!10] (Triangle12) [below of=OF12,yshift=-1.5em] {Háromszögelés};

\draw (OF12) -- (Triangle12);

\node[box] (Triangle23) [below of=OF23,yshift=-1.5em] {Háromszögelés};

\draw (OF23) -- (Triangle23);


\node[plain] (Others) [below of=FgMask3,xshift=10em,yshift=-9em,text width=2em] {$\vdots$};

\node[box] (Merge) [below of=Triangle12,xshift=8em,yshift=-2.5em] {Összefűzés};

\draw (Triangle12) |- (Merge);
\draw (Triangle23) |- (Merge);
\draw (Others) |- (Merge);


\node[box,fill=green!10] (Contour) [below of=Merge,yshift=-1.5em] {Nézőpontból\\ projekció + kontúr};

\draw (Merge) -- (Contour);



\begin{pgfonlayer}{background}
    \path (Cam1.north west)+(-0.5,0.5) node (a) {};
    \path (FgMask1.south east)+(+0.5,-0.5) node (b) {};
    \path[rounded corners, draw=black!50, dashed] (a) rectangle (b);
\end{pgfonlayer}

\begin{pgfonlayer}{background}
    \path (Cam2.north west)+(-0.5,0.5) node (a) {};
    \path (FgMask2.south east)+(+0.5,-0.5) node (b) {};
    \path[rounded corners, draw=black!50, dashed] (a) rectangle (b);
\end{pgfonlayer}

\begin{pgfonlayer}{background}
    \path (Cam3.north west)+(-0.5,0.5) node (a) {};
    \path (FgMask3.south east)+(+0.5,-0.5) node (b) {};
    \path[rounded corners, draw=black!50, dashed] (a) rectangle (b);
\end{pgfonlayer}


\begin{pgfonlayer}{background}
    \path (CamN.north west)+(-0.5,0.5) node (a) {};
    \path (FgMaskN.south east)+(+0.5,-0.5) node (b) {};
    \path[rounded corners, draw=black!50, dashed] (a) rectangle (b);
\end{pgfonlayer}

\end{tikzpicture}
}

\caption{Elkészített megoldás átfogó terve \label{fig:of-method}}
\end{sidewaysfigure}

% --------------------------------
\section{Összefoglaló}
% --------------------------------

Ebben a fejezetben bemutattam az alkalmazott keretrendszert, melyre a megoldásomat építettem. Kidolgoztam a specifikációt, miszerint két rögzített kamera által megfigyelt térrészben rekonstruálok egy illetve két mozgó objektumot egy a két kamera közötti szakaszon választott nézőpontból. Rögzítettem a sikerkritériumot, hogy a helyreállítást akkor tekintem sikeresnek, ha a mozgó objektumok megközelítő kontúrjait sikeresen meghatározom. Végül leírtam az implementálásra kerülő rendszer lépésekre bontott logikáját.