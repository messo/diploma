%----------------------------------------------------------------------------
\chapter*{Bevezetés}\addcontentsline{toc}{chapter}{Bevezetés}
%----------------------------------------------------------------------------

Az informatika, ezen belül pedig a gépi látással foglalkozó terület, valamint az ehhez szükséges számítási kapacitások, célhardverek rohamos fejlődésével, mind újabb, hatékonyabb valamint pontosabb megoldások születtek és születnek a felmerült problémákra.

Míg egy-két évtizeddel ezelőtt a gépi látáshoz kapcsolódó kutatások jelentős részét főként a robotika, a katonság (pl. drónok), valamint az űrkutatás adta, manapság már a mindennapi élet gyökeres részévé vált. Vegyük például a közlekedést; a közép-felső kategóriás autóknál már tulajdonképpen szériatartozéknak tekinthető az elülső és hátsó tolató radar. Ugyanígy a kereskedelmi forgalomban kapható kis robot-porszívók is rendelkeznek beépített kamera/radar-rendszerrel, amely a beltéri navigációt segíti. A napjainkban kapható játékkonzolokhoz is vásárolható kiegészítő kamera rendszer, mely a játékos mozgását, annak elhelyezkedését figyeli, lényegében a játékos a saját testét használja vezérlőként. Szórakozást tekintve a manapság egyre nagyobb teret kapó quadcoptereket \cite{quadropter} is említeni kell, ezek is rendelkeznek kamerával (vagy rájuk szerelhető), és már folynak kutatások, amelyek ezek akár autonóm \cite{quad-autonomous}, akár tömeges \cite{quad-swarm} -- rajban történő -- repülését vizsgálja.

A dolgozat motivációját a következő feltevésekhez hasonló problémák adták:
\begin{itemize}
\item Egy izgalmas gólhelyzet során mit láthatott a kapus a kapuban állva?
\item Egy bankrablási szituációban mit láthatott az elkövető, és mit a biztonsági őr?
\item Egy vizsga során, a gyanús egyetemista láthatta-e az előtte ülő dolgozatát?
\item Egy közlekedési balesetben mit láthatott a biciklis, és mit a buszsofőr?
\end{itemize}

Ezekre és ehhez hasonló kérdésekre választ nyújthat a kitűzött feladat megoldása, miszerint több kamerával megfigyelt térrészt egy választott nézpontból rekonstruálunk. Hiszen gondoljuk meg, hogy a fenti esetekben nem oldható meg, hogy a választott személy nézőpontjába valódi kamerákat állítsunk.

Az 1. fejezetben a feladat megoldásához szükséges elméleti háttérről lesz szó, mely tartalmazza a háromdimenziós tér alapvető transzformációit, valamint a dolgozat során használt kameramodellt.

A 2. fejezetben a több kamerából álló kamerarendszereknél felmerülő problémák kerülnek tárgyalásra, illetve, hogy ezekre milyen, a napjainkban is használt megoldások léteznek, kitérve a feladat során szükséges részproblémák megoldására is.

A 3. fejezetben bemutatásra kerül az előző fejezetek által leírt információk és algoritmusok alapján egy, a kitűzött feladatra megoldást adó rendszer váza, megvalósítási terve. %, és a tervezett rendszerrel szemben támasztott követelmények.
