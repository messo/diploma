%----------------------------------------------------------------------------
\chapter*{Bevezetés}\addcontentsline{toc}{chapter}{Bevezetés}
%----------------------------------------------------------------------------

Az informatika, ezen belül pedig a gépi látással foglalkozó terület, valamint az ehhez szükséges számítási kapacitások, célhardverek rohamos fejlődésével, mind újabb, hatékonyabb valamint pontosabb megoldások születtek és születnek a felmerült problémákra.\\

Míg 1-2 évtizeddel ezelőtt a gépi látáshoz kapcsolódó kutatások jelentős részét főként a robotika, a katonság (pl. drónok), valamint az űrkutatás adta, manapság már a mindennapi élet gyökeres részévé vált. Vegyük például a közlekedést; a közép-felső kategóriás autóknál már szinte széria tartozék az elülső és hátsó tolató radar. Ugyanígy a kereskedelmi forgalomban kapható kis robot-porszívók is rendelkeznek beépített kamera/radar-rendszerrel, amely a beltéri navigációt segíti. A manapság kapható játék-konzolokhoz is kapható kiegészítő kamera-rendszer, mely a játékos mozgását, annak elhelyezkedését figyeli, lényegében a játékos a saját testét használja vezérlőként. Szórakozást tekintve, a manapság egyre nagyobb teret kapó quadroptereket \cite{quadropter} is említeni kell, ezek is rendelkeznek vagy rájuk szerelhető kamera, és már folynak kutatások, amelyek ezek autonóm, akár tömeges repülését kutatja.\\

A dolgozat motivációját a következő feltevésekhez hasonló problémák adták:
\begin{itemize}
\item egy izgalmas gólhelyzet során, mit látott a kapus a kapuban állva?
\item egy bankrablási szituációban, mit látott az elkövető és mit a betörő?
\item egy vizsga során, a gyanús egyetemista láthatta-e az előtte ülő dolgozatát?
\item egy közelekedési balesetben, mit láthatott a biciklis és mit a buszsofőr?
\end{itemize}

Ezekre és ehhez hasonló kérdésekre választ nyújthat a kitűzött feladat, mi szerint több kamerával megfigyelt térrészt egy választott nézpontból rekonstruáljuk. Hiszen a fenti esetekben nem oldható meg, hogy a választott személy nézőpontjába valódi kamerákat állítsunk (a fejra/vállra szerelt kamera se az igazi, hiszen egy kicsit más nézőpontot ad).\\


Az 1. fejezetben a feladat megoldásához szükséges elméleti részről lesz szó, mely tartalmazza a 3 dimenziós tér alapvető transzformációit valamint a dolgozat során használt kameramodellt.

A 2. fejezetben szó lesz a több kamerából álló kamerarendszereknél felmerülő problémákra, illetve, hogy ezekre milyen, a napjainkban is használt megoldások léteznek, kitérve a feladat során szükséges részproblémák megoldására is.

A 3. fejezetben bemutatásra kerül, az előző fejezetek által leírt információk és algoritmusok alapján, egy a kitűzött feladatra megoldást adó rendszer váza, megvalósítási terve, és a tervezett rendszerrel szemben támasztott követelmények.
