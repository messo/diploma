%----------------------------------------------------------------------------
\chapter{Összefoglalás}
%----------------------------------------------------------------------------

A munka első felét elvégeztem, áttekintettem a szükséges irodalmat, elméleti hátteret, valamint megterveztem a következő félévben implementálandó rendszert.

Az 1. fejezetben az alapvető háromdimenziós transzformációkat mutattam be a szükséges lineáris algebra bevezetőt követően, valamint megadtam a szakirodalomban és a dolgozat során is használt lyukkamera-modellt, mely a kép- és videofeldolgozási algoritmusok alapjául szolgál.

A 2. fejezetben a több kamerából álló rendszereknél felmerülő, a feladatom során is megoldandó problémákat jártam körbe. Kitértem a kamerák szinkronizációjára, valamint a kamerák által látott képekből a háromdimenziós világ lehetséges rekonstrukciójára. Ehhez két lényegében eltérő megoldást vázoltam és mutattam be; a sztereó-kalibrációt valamint az optikai-folyamokat. A fejezetet az objektum-detekcióval zártam, mely felhasználásával a probléma megoldásának keresési tere csökkenthető.

A 3. fejezetben az előző fejezetekben megismert megközelítésekre és algoritmusokra támaszkodva megterveztem a következő félévben elkészítendő rendszert, kitérve a két eltérő megközelítés közti hasonlóságokra és különbségekre.