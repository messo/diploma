%----------------------------------------------------------------------------
\chapter{Tervezés}
%----------------------------------------------------------------------------

Az elméleti megfontolások, alapelvek bemutatását követően rátérek a feladat megoldását adó rendszer struktúrájának, főbb komponenseinek tárgyalására.

\section{Keretrendszer}

A diplomamunka során az OpenCV \cite{opencv} keretrendszert fogom használni melynek célja, hogy a fejlesztőknek, egy szabadon és ingyenes elérhető alkalmazás-könyvtárat biztosítson a gépi látás területén elterjedt és gyakran használt algoritmusokhoz. Több nyelvhez is biztosít API-t, én ezek közül a C++-os interfészét fogom használni, a lehető legnagyobb teljesítmény elérése érdekében. Az OpenCV-t már az előző félévekben megismertem, így a diplomamunka során már eredményesen tudok tudok építekezni az általa nyújtott funkciókra.

\section{A konkrét elvárt eredmény}

A diplomamunka során elkészítendő rendszernek képesnek kell lennie egy zárt térrész tetszőlegesen választott pontjában és irányában látható, a mozgó objektumokat tartalmazó kép helyreállítására fix telepítésű kamerák valós idejű videofolyamai alapján. A helyreállítást akkor tekintjük sikeresnek, ha a mozgó objektumokat detektálom és a választott nézőpontból azok megközelítő kontúrjait sikeresen ki tudom rajzolni. A tényleges, valódi objektum struktúrájának illetve textúrájának meghatározása nem esik e dolgozat hatáskörébe.

A rendszer tervezése során figyelembe veszem, hogy kettőnél több kamerát is felhasználhatunk a probléma megoldásához, de az elkészült rendszer az elérhető eszközök korlátozott száma miatt csak két kamerával lesz kipróbálva. A rendszer teljesítményére vonatkozóan viszont következtetéseket vonok le, hogy több kamera bevonása esetén, a számítási kapacitás növelése nélkül milyen csökkenésre lehet számítani.

\section{Tervezett rendszer felépítése}

Ebben a fejezetben az elméleti rész során \aref{sec:methods}. szekcióban leírt kétféle megközelítéshez tartozó rendszerek terveit mutatom be.

\subsection{Sztereó-látáson alapú megközelítés}

\begin{figure}[t]
\centering

\pgfdeclarelayer{background}
\pgfdeclarelayer{foreground}
\pgfsetlayers{background,main,foreground}

\begin{tikzpicture}[->,>=stealth',shorten >=1pt,auto]
\tikzset{
box/.style={draw, rectangle, text width=12em, minimum height=2.5em, text centered},
plain/.style={text width=12em, text centered},
line/.style = {-,shorten >=0pt},
graybox/.style = {box, gray}
}

\node[plain] (Cam12) {1. kamera pár};

\node[graybox] (SCalib) [below of=Cam12] {Sztereó kalibrálás};

\node[box] (GetFrames1) [below of=SCalib,yshift=-1.5em] {Aktuális képkockák lekérése};

\node[box] (Rectify) [below of=GetFrames1,yshift=-1.5em] {Rektifikáció};

\node[box] (FwdBkg) [below of=Rectify,yshift=-1.5em] {Előtér-háttér szegmentáció};

\node[box] (Depth) [below of=FwdBkg,yshift=-1.5em] {Mélység-kép meghatározása};

\node[box] (3D) [below of=Depth,yshift=-1.5em] {Pontfelhő rekonstrukció};

\draw (SCalib) -- (GetFrames1);
\draw (GetFrames1) -- (Rectify);
\draw (Rectify) -- (FwdBkg);
\draw (FwdBkg) -- (Depth);
\draw (Depth) -- (3D);

\node[plain] (LDots) [right of=Rectify,xshift=7.75em] {$\ldots$};

\node[plain] (Cam34) [right of=Cam12,xshift=18em] {$n$. kamera pár};

\node[graybox] (SCalib2) [below of=Cam34] {Sztereó kalibrálás};

\node[box] (GetFrames2) [below of=SCalib2,yshift=-1.5em] {Aktuális képkockák lekérése};

\node[plain] (Dots1) [below of=GetFrames2,yshift=-1.5em] {$\vdots$};

\node[box] (3D2) [right of=3D,xshift=18em] {Pontfelhő rekonstrukció};

\node[plain] (Dots2) [right of=Depth,xshift=18em] {$\vdots$};

\draw (SCalib2) -- (GetFrames2);
\draw (GetFrames2) -- (Dots1);
\draw (Dots2) -- (3D2);

\node[box] (Coords) [below of=3D,xshift=10.35em,yshift=-2.5em] {Koordináta-rendszer egységesítés};

\draw (3D) |- (Coords);
\draw (3D2) |- (Coords);

\node[box] (Viewpoint) [below of=Coords,yshift=-1.5em] {Választott nézőpontból kontúrok};

\draw (Coords) -- (Viewpoint);

\coordinate (Bottom) at ($(Viewpoint) + (0,-3em)$);

\draw (Viewpoint) -- (Bottom);

\draw (Bottom) -- ++(-7.5,0) |- (GetFrames1);
\draw (Bottom) -- ++(7.5,0) |- (GetFrames2);

\begin{pgfonlayer}{background}
    \path (Cam12.north west)+(-0.5,0.5) node (a) {};
    \path (3D.south east)+(+0.5,-0.5) node (b) {};
    \path[rounded corners, draw=black!50, dashed] (a) rectangle (b);
\end{pgfonlayer}

\begin{pgfonlayer}{background}
    \path (Cam34.north west)+(-0.5,0.5) node (a) {};
    \path (3D2.south east)+(+0.5,-0.5) node (b) {};
    \path[rounded corners, draw=black!50, dashed] (a) rectangle (b);
\end{pgfonlayer}

\end{tikzpicture}

\caption{Sztereókamerák módszere \label{fig:stereo-method}}
\end{figure}

Az első megközelítés kiinduló pontja, hogy a kamerák az előző fejezetben leírt módszerekkel páronként sztereókalibráltak. Ezután ciklusban amíg a program fut, a bejövő képkockákat kamera-pároként feldolgozza a következők szerint. Először rektifikálja a bejövő képet, hogy a mélységképet megkaphassuk. A folyamatos háttérmodell-építésnek köszönhetően mindig lekérhetjük az aktuális előtérnek megfelelő maszkot, ezt alkalmazzuk a képeken, így csak a számunkra érdekes képinformáció marad meg. Az eredményen a mélységképet kiszámoljuk az SGBM algoritmus segítségével és így megkaphatjuk a pontfelhőket. Amikor minden kamerapárra az előzőek lefutottak, akkor az összegyűjtött eredményeket egy közös koordináta-rendszerbe transzformáljuk. A választott nézőpontunkból ezután a pontfelhőhöz megrajzoljuk az objektumok kontúrjait és így az adott pillanat rekonstrukcióját kapjuk. Az előzőekben leírt lépéseket \aref{fig:stereo-method}. ábra mutatja be. 

\subsection{Optikai folyamokon alapuló megközelítés}

E módszer lényege, hogy a kamerákat külön-külön használjuk, csak azok egyéni képei alapján rekonstruáljuk a mozgó objektumokat az adott nézőpontokból. Az optikai folyamok segítségével képről képre megbecsülhetjük, hogy az objektum pontjai hova mozdultak, és ezeket felhasználva a háromszögeléses módszer segítségével meghatározhatjuk az objektum közelítő pontfelhőjét. Ezeket, hasonlóan az előző megközelítéshez, egy közös koordináta-rendszerben ábrázoljuk, végül a választott nézőpontból a látott objektumokhoz a kontúrokat megrajzoljuk az aktuális pillanathoz tartozóan lásd \aref{fig:of-method}. ábrát.

\begin{figure}[tbh]
\centering

\pgfdeclarelayer{background}
\pgfdeclarelayer{foreground}
\pgfsetlayers{background,main,foreground}

\begin{tikzpicture}[->,>=stealth',shorten >=1pt,auto]
\tikzset{
box/.style={draw, rectangle, text width=12em, minimum height=2.5em, text centered},
plain/.style={text width=12em, text centered},
line/.style = {-,shorten >=0pt},
graybox/.style = {box, gray}
}

\node[plain] (Cam12) {1. kamera};

\node[graybox] (SCalib) [below of=Cam12] {Kalibrálás};

\node[box] (GetFrames1) [below of=SCalib,yshift=-1.5em] {Aktuális képkocka lekérése};

\node[box] (OF1) [below of=GetFrames1,yshift=-1.5em] {Optikai folyam meghatározása};

\node[box] (Triangulation) [below of=OF1,yshift=-1.5em] {Háromszögelés};

\node[box] (CloudUpdate) [below of=Triangulation,yshift=-1.5em] {Pontfelhő frissítése az új adatokkal};

\draw (SCalib) -- (GetFrames1);
\draw (GetFrames1) -- (OF1);
\draw (OF1) -- (Triangulation);
\draw (Triangulation) -- (CloudUpdate);

\node[plain] (LDots) [right of=OF1,xshift=7.75em] {$\ldots$};

\node[plain] (Cam34) [right of=Cam12,xshift=18em] {$n$. kamera};

\node[graybox] (SCalib2) [below of=Cam34] {Kalibrálás};

\node[box] (GetFrames2) [below of=SCalib2,yshift=-1.5em] {Aktuális képkocka lekérése};

\node[plain] (Dots1) [below of=GetFrames2,yshift=-1.5em] {$\vdots$};

\node[plain] (Dots2) [right of=Triangulation,xshift=18em] {$\vdots$};

\node[box] (CloudUpdate2) [right of=CloudUpdate,xshift=18em] {Pontfelhő frissítése az új adatokkal};

\draw (SCalib2) -- (GetFrames2);
\draw (GetFrames2) -- (Dots1);
\draw (Dots2) -- (CloudUpdate2);

\node[box] (Coords) [below of=CloudUpdate,xshift=10.35em,yshift=-2.5em] {Koordináta-rendszer egységesítés};

\draw (CloudUpdate) |- (Coords);
\draw (CloudUpdate2) |- (Coords);

\node[box] (Viewpoint) [below of=Coords,yshift=-1.5em] {Választott nézőpontból kontúrok};

\draw (Coords) -- (Viewpoint);

\coordinate (Bottom) at ($(Viewpoint) + (0,-3em)$);

\draw (Viewpoint) -- (Bottom);

\draw (Bottom) -- ++(-7.5,0) |- (GetFrames1);
\draw (Bottom) -- ++(7.5,0) |- (GetFrames2);

\begin{pgfonlayer}{background}
    \path (Cam12.north west)+(-0.5,0.5) node (a) {};
    \path (CloudUpdate.south east)+(+0.5,-0.5) node (b) {};
    \path[rounded corners, draw=black!50, dashed] (a) rectangle (b);
\end{pgfonlayer}

\begin{pgfonlayer}{background}
    \path (Cam34.north west)+(-0.5,0.5) node (a) {};
    \path (CloudUpdate2.south east)+(+0.5,-0.5) node (b) {};
    \path[rounded corners, draw=black!50, dashed] (a) rectangle (b);
\end{pgfonlayer}

\end{tikzpicture}

\caption{Optikai folyamok módszere \label{fig:of-method}}
\end{figure}