%----------------------------------------------------------------------------
% Abstract in hungarian
%----------------------------------------------------------------------------
\chapter*{Kivonat}\addcontentsline{toc}{chapter}{Kivonat}

Több évtizede folynak kutatások a gépi látás területén. A technika és a számítási kapacitás növekedésével, minél újabb és hatékonyabb algoritmusok látnak napvilágot, melyek a felmerült problémák különböző részhalmazait oldják meg. Napjainkban több különböző területen is támaszkodunk ezen megoldásokra, legyen az egészségügy, biztonság, szórakozás vagy éppen a kényelmünk biztosítása.

A diplomaterv célja, hogy megvizsgálja egy fix telepítésű kamerákkal megfigyelt térrészben egy választott pontban és irányban látható kép valós idejű helyreállíthatóságát.

Első feladatom, hogy áttekintsem és megismerjem a háromdimenziós tér alapvető transzformációit, valamint a több kamerából álló rendszerek sajátosságait, azoknál felhasználható elméletet és gyakorlati megközelítéseket. Ezt követően egy, a kamerák által megfigyelt térrészben mozgó objektumokat tartalmazó kép egy választott pontból történő rekonstrukcióját vizsgálom meg. Végül bemutatom, az általam tervezett alkalmazás vázát, mely elvégzi a rekonstrukciót, és választ ad arra, hogy milyen feltételek és körülmények között tekinthető a megoldás valós idejűnek.

\vfill

%----------------------------------------------------------------------------
% Abstract in english
%----------------------------------------------------------------------------
%\chapter*{Abstract}\addcontentsline{toc}{chapter}{Abstract}

%This document is a \LaTeX-based skeleton for BSc/MSc~theses of students at the Electrical Engineering and Informatics Faculty, Budapest University of Technology and Economics. The usage of this skeleton is optional. It has been tested with the \emph{TeXLive} \TeX~implementation, and it requires the PDF-\LaTeX~compiler.
%\vfill

